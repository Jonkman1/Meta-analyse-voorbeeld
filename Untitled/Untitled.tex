% Options for packages loaded elsewhere
\PassOptionsToPackage{unicode}{hyperref}
\PassOptionsToPackage{hyphens}{url}
% !TeX program = pdfLaTeX
\documentclass[12pt]{article}
\usepackage{amsmath}
\usepackage{graphicx,psfrag,epsf}
\usepackage{enumerate}
\usepackage[]{natbib}
\usepackage{textcomp}


%\pdfminorversion=4
% NOTE: To produce blinded version, replace "0" with "1" below.
\newcommand{\blind}{0}

% DON'T change margins - should be 1 inch all around.
\addtolength{\oddsidemargin}{-.5in}%
\addtolength{\evensidemargin}{-1in}%
\addtolength{\textwidth}{1in}%
\addtolength{\textheight}{1.7in}%
\addtolength{\topmargin}{-1in}%

%% load any required packages here



% tightlist command for lists without linebreak
\providecommand{\tightlist}{%
  \setlength{\itemsep}{0pt}\setlength{\parskip}{0pt}}




\IfFileExists{bookmark.sty}{\usepackage{bookmark}}{\usepackage{hyperref}}
\IfFileExists{xurl.sty}{\usepackage{xurl}}{} % add URL line breaks if available
\hypersetup{
  pdftitle={`De appel valt niet ver van de boom'},
  pdfkeywords={adverse childhood
experiences,, sensitivity,, intergenerational
transmission,, metaanalysis},
  hidelinks,
  pdfcreator={LaTeX via pandoc}}



\begin{document}


\def\spacingset#1{\renewcommand{\baselinestretch}%
{#1}\small\normalsize} \spacingset{1}


%%%%%%%%%%%%%%%%%%%%%%%%%%%%%%%%%%%%%%%%%%%%%%%%%%%%%%%%%%%%%%%%%%%%%%%%%%%%%%

\if0\blind
{
  \title{\bf `De appel valt niet ver van de boom'}

  \author{
        A.M. Hovestadt \thanks{H. Jonkman} \\
    Graduate School of Child Development and Education, Universiteit van
Amsterdam (Masterscriptie)\\
      }
  \maketitle
} \fi

\if1\blind
{
  \bigskip
  \bigskip
  \bigskip
  \begin{center}
    {\LARGE\bf `De appel valt niet ver van de boom'}
  \end{center}
  \medskip
} \fi

\bigskip
\begin{abstract}
Children of parents with adverse childhood experiences (ACEs) are more
likely to experience ACEs themselves through intergenerational
transmission. The purpose of this meta-analysis with a random-effect
model was to study the effect of parents with ACEs on their sensitivity
towards their children. Twelve studies with one effect size each were
included. There appeared to be heterogeneity between studies
(\(Q (11) = 29.13, p = .002\)), which allowed moderator analyses. A
significant middle size mean effect was found of parents' ACEs on
sensitivity towards their children
(\(r = -.218, 95% CI = [-.293, -0.140]
\)). This result confirmed the hypothesis that parental ACEs resulted in
less sensitivity towards their children. The number of ACEs, age of
parents, nor age of children were found to have significant moderation
effects. Follow-up research is needed to examine whether the results
apply to both fathers and mothers, and to see if the type of ACEs
influences the relationship.
\end{abstract}

\noindent%
{\it Keywords:} adverse childhood
experiences,, sensitivity,, intergenerational
transmission,, metaanalysis

\vfill

\newpage
\spacingset{1.9} % DON'T change the spacing!

\section{Inleiding}\label{inleiding}

`De appel valt niet ver van de boom' is een bekend spreekwoord dat in
het Algemeen Nederlands Woordenboek (z.d.) omschreven wordt als
`kinderen lijken vaak op hun ouders'. In sommige gevallen is dat
positief bedoelt, maar er zijn ook gevallen waar dit niet zo is. Uit de
laatste meting van kindermishandeling in Nederland bleek dat 96\% van de
mishandelde kinderen is mishandeld door een van de biologische ouders
(Alink et al., 2018). Het meemaken van kindermishandeling is een van de
ingrijpendste jeugdervaringen die je als kind kan meemaken. Kinderen die
zelf mishandeld zijn door hun ouders hebben een grotere kans om hun
eigen kinderen later te mishandelen. Uit onderzoek kwam naar voren dat
40\% van de ouders die zijn opgegroeid in een onveilige situatie, ook
hun eigen kinderen in een onveilige situatie brengen (Eimers, 2021). Dit
werd bevestigd in een ander onderzoek waarin werd aangetoond dat
kinderen van moeders met een geschiedenis van mishandeling tijdens de
kindertijd, meer dan twee keer zoveel risico liepen om
kindermishandeling te ervaren (Bérubé et al., 2020). Wanneer het
doorbreken van het patroon niet lukt bij ouders die ingrijpende
jeugdervaringen hebben meegemaakt, is er sprake van intergenerationele
overdracht van ingrijpende jeugdervaringen.\\
Ingrijpende jeugdervaringen, in het Engels Adverse Childhood Experiences
(ACE's) genoemd, zijn traumatische gebeurtenissen die plaatsvinden in de
eerste achttien levensjaren (Felitti et al., 1998). In totaal zijn er
tien soorten ingrijpende jeugdervaringen, onderverdeeld in drie
categorieën: misbruik, verwaarlozing en disfunctioneren van het
huishouden. Onder misbruik vallen: fysiek, emotioneel en seksueel
misbruik. Onder de categorie verwaarlozing vallen zowel fysieke als
emotionele verwaarlozing. Tot slot vallen onder de categorie
disfunctioneren van het huishouden: echtscheiding van ouders, gezinslid
met middelenmisbruik, gezinslid met psychische aandoening, opgesloten
familielid en getuige zijn van huiselijk geweld (Boullier \& Blair,
2018).\\
Felitti en collega's (1998) hebben baanbrekend onderzoek gedaan naar de
impact van ingrijpende jeugdervaringen op toekomstige medische
gezondheid en welzijn. Er werd in dit onderzoek een trapsgewijs verband
gevonden tussen het aantal ingrijpende jeugdervaringen en een verhoogde
kans op een gebrekkige gezondheid op latere leeftijd. Het meemaken van
ingrijpende jeugdervaringen heeft schadelijke medische gevolgen en heeft
een negatieve invloed op de sociaal-emotionele ontwikkeling (Felitti et
al., 1998). Uit onderzoeken kwam naar voren dat ingrijpende
jeugdervaringen gepaard kunnen gaan met verstoorde hechting,
gedragsproblemen en depressie op latere leeftijd (Alto et al., 2021;
Flagg et al., 2023).\\
Hoe vaker een kind ingrijpende jeugdervaringen heeft meegemaakt, hoe
nadeliger de gevolgen voor zijn/haar toekomst (Felitti et al., 1998).
Uit onderzoeken blijkt dat meer dan 50\% van de mensen één ingrijpende
jeugdervaring heeft meegemaakt en 10\% van de mensen drie of meer
ingrijpende jeugdervaringen (Felitti et al., 1998; Ziv et al., 2018). De
blootstelling aan vier of meer ingrijpende jeugdervaringen is een
hardnekkige risicofactor die gevolgen kan hebben in het verdere leven
(Felitti et al., 1998).\\
Het meemaken van ingrijpende jeugdervaringen heeft niet alleen invloed
op de medische gezondheid en de sociaal-emotionele ontwikkeling, maar
ook op de kwaliteit van het ouderschap (Friesen et al., 2017; Olhaberry
et al., 2022; Savage et al., 2019; Zvara et al., 2015). Er wordt al
langere tijd gedacht dat de aard en kwaliteit van ouderschap
intergenerationeel wordt overgedragen op de volgende generatie (Flagg et
al., 2023). Deze gedachte wordt ondersteund door de levenslooptheorieën
(Elder, 1981), hechtingstheorieën (Bowlby, 1969) en theorieën over
sociaal leven (Patterson, 1998). Een kind dat een ingrijpende ervaring
meemaakt, heeft een grotere kans om niet de vaardigheden te ontwikkelen
om op een adequate manier met uitdagende situaties om te gaan. Dit kan
er vervolgens voor zorgen dat wanneer het kind zelf opvoeder wordt, het
niet de vaardigheden heeft om zijn/haar eigen kind op een adequate wijze
op te voeden (Brodsky, 2016).\\
Een van die vaardigheden van het ouderschap is om als ouder emotioneel
beschikbaar te zijn voor je kind. Een van de vier oudercomponenten die
onder emotionele beschikbaarheid valt, is sensitief zijn (MacMillan et
al., 2021). Een sensitieve ouder heeft het vermogen om de emotionele
behoeften van zijn/haar kind te herkennen. Een onderdeel van
sensitiviteit is responsiviteit, wat wil zeggen dat er adequaat wordt
gereageerd op de emotionele behoefte van het kind (Clark et al., 2021).
Volgens Ainsworth en collega's (1974) is sensitiviteit een cruciaal
onderdeel voor de hechting tussen ouder en kind. Sensitiviteit zorgt er
namelijk voor dat een kind zich kan exploreren en terug kan vallen op
ouders als het kind ondersteuning nodig heeft. Gegeven het belang van
sensitiviteit van ouders is het belangrijk te begrijpen welke factoren
hier een rol op kunnen spelen. Een van deze factoren is het effect van
ingrijpende jeugdervaringen.\\
Hoewel in meerdere onderzoeken duidelijk werd dat de kwaliteit van
ouderschap van generatie op generatie wordt doorgegeven, moet vermeld
worden dat er ook sprake van intergenerationele discontinuïteit kan
zijn, wat betekent dat deze overdracht niet plaatsvindt (Flagg et al.,
2023; Rutter, 1998; Zvara et al., 2017). Over het onderliggende
mechanisme dat ervoor zorgt of er wel of geen sprake is van
intergenerationele overdracht van ingrijpende jeugdervaringen is nog
weinig bekend (Narayan et al., 2021; Wattanatchariya et al., 2024). In
sommige onderzoeken wordt dit mechanisme wel besproken, maar in de
meeste gevallen is alleen mishandeling als ingrijpende jeugdervaring
meegenomen en zijn de andere ingrijpende jeugdervaringen buiten
beschouwing gelaten (Rowell \& Neal‐Barnett, 2021). Daarnaast wordt
sensitiviteit in de meeste onderzoeken niet als losstaande variabele
onderzocht, maar als moderator of als onderdeel van hechting (Bérubé et
al., 2024).\\
Samengevat ontbreekt er onderzoek naar het verband tussen alle tien de
vormen van de ingrijpende jeugdervaringen van ouders op de sensitiviteit
naar hun kinderen toe en ontbreekt er duidelijkheid over welke
onderliggende mechanismen effect hebben op dit verband. Het meemaken van
meerdere vormen van ingrijpende jeugdervaringen kan zorgen voor nadelige
gevolgen zoals depressie, een verminderde kwaliteit van ouderschap en
sociaal-emotionele ontwikkeling, maar het is niet duidelijk of hier
vandaag de dag nog sprake van is. Het doel van dit onderzoek was om een
duidelijker beeld te krijgen van het verband tussen ingrijpende
jeugdervaringen bij ouders en de sensitiviteit naar hun kinderen toe. In
dit onderzoek werd de focus gelegd op de richting van het verband tussen
deze variabelen, de sterkte van dit verband en de invloed van een
drietal moderatoren op dit verband: het aantal ingrijpende
jeugdervaringen, de leeftijd van ouders ten tijde van het onderzoek en
de leeftijd van het kind ten tijde van het onderzoek. Alle soorten
ingrijpende jeugdervaringen zijn meegenomen tijdens dit onderzoek. De
eerste hypothese was dat er een negatief verband is tussen de
onderzochte variabelen: ingrijpende jeugdervaringen bij ouders zorgen
ervoor dat ze minder sensitief zijn naar hun kinderen toe. De tweede
hypothese was dat hoe meer ingrijpende jeugdervaringen ouders hebben
meegemaakt, hoe minder sensitief ouders zijn. De derde hypothese was dat
jongere ouders minder sensitief zijn naar hun kinderen toe doordat zij
minder tijd hebben gehad om de ingrijpende jeugdervaringen te verwerken.
De vierde en laatste hypothese was dat ouders van jongere kinderen
minder sensitief zijn vanwege het gebrek aan ervaring en omdat jongere
kinderen een groter beroep doen op ouders.\\
\#\# Methode\\
\#\#\# Inclusie- en exclusiecriteria\\
Aan het begin van de meta-analyse zijn inclusie- en exclusiecriteria
opgesteld waaraan de geïncludeerde studies moesten voldoen. Ten eerste
werden er alleen primaire studies meegenomen die minimaal één
ingrijpende jeugdervaring bij ouders als onafhankelijke variabele
onderzochten en ouderschap als afhankelijke variabele onderzochten. Ten
tweede werden alleen studies meegenomen van ouders met kinderen tussen
de 0-18 jaar. Ten derde werden alleen peer reviewed studies meegenomen
om de kwaliteit van de studies tot zekere hoogte te garanderen. Ten
vierde werden alleen studies meegenomen die zijn gepubliceerd tussen
1998 en 2024, aangezien het Engelse begrip ACE's pas in 1998 is bedacht
(Felitti et al., 1998). Tot slot werden alleen Engelstalige studies
meegenomen die zijn uitgevoerd in westerse landen. Exclusiecriteria
waren meta-analyses, proefschriften, artikelen geschreven in een andere
taal dan het Engels en studies die zijn uitgevoerd in niet-westerse
landen.\\

\subsection{Zoekstrategie}\label{zoekstrategie}

Er werd op systematische wijze gezocht naar gepaste studies in
verschillende databases. De zoekstrings zijn opgesteld in samenwerking
met een Specialist Wetenschappelijke Informatie Psychologie, Pedagogiek
en Onderwijskunde werkzaam bij de Universiteit van Amsterdam. De
zoekstrings werden gebruikt om informatie te vinden over ingrijpende
jeugdervaringen, ouderschap en uitkomstmaten\footnote{De gebruikte
  zoekstrings kunnen worden opgevraagd bij de studiebegeleider.}. Er
werd gezocht in de databases PsychInfo, Medline en Web of Science Core
Collection. In totaal werden 20.493 studies gevonden, waarvan in Zotero
één databestand werd gemaakt (Versie 6.0.36; 2024). Vervolgens werden
via het programma DedupEndNote (Lobbestael, 2023) de duplicaten
verwijderd. De overgebleven artikelen werden verdeeld in vier datasets,
waarbij elke dataset door twee beoordelaars werden gescreend op basis
van hun abstract in ASReview (Versie 1.5; ASReview LAB developers,
2023). ASReview is een AI-screeningstool dat artikelen op volgorde van
relevantie sorteert aan de hand van de beoordeling van de onderzoeker,
waardoor tijd bespaard kan worden. Tijdens dit proces werd een stopregel
gehanteerd van 100 opeenvolgende irrelevante artikelen. Bij deze eerste
screening werd gelet of ingrijpende jeugdervaringen als afhankelijke
variabele en ouderschap als onafhankelijke variabelen onderzocht werden.
Na het opstellen van de onderzoeksvraag zijn de overgebleven artikelen
opnieuw gescreend in ASReview waarbij ditmaal gelet werd of
sensitiviteit als afhankelijke variabele was meegenomen. Van de
overgebleven 233 artikelen werd de gehele tekst gescreend en werden de
artikelen geselecteerd die voldeden aan alle inclusiecriteria en
daardoor geschikt waren voor huidig onderzoek. Uiteindelijk zijn er
twaalf studies meegenomen in de analyses (zie Figuur 1).

Screenshot 2 (Figuur1: Stroomdiagram van zoekresultaten volgens de
Preferred Reporting Items for Systematic Review and MetaAnalysis
(PRISMA))

\subsection{Coderen}\label{coderen}

Een codeerschema werd ontworpen in Microsoft Excel om relevante gegevens
uit de artikelen te coderen. Om een helder beeld te krijgen van de
ingrijpende jeugdervaringen die zijn meegenomen in de studies is er
gecodeerd bij welke ouder(s) de ingrijpende jeugdervaringen gemeten
werden (moeder en/of vader), welke typen ingrijpende jeugdervaringen
werden gemeten en hoeveel ingrijpende jeugdervaringen (1-10) er gemeten
werden. Daarnaast werd er verschillende informatie over de
steekproefpopulatie gecodeerd: de steekproefgrootte, de gemiddelde
leeftijd van de ouder (\textless20, 20-25, 26-30, 31-35, \textgreater35)
en de gemiddelde leeftijd van het kind (\textless6 mnd, 7-12 mnd, 1-4
jaar, 5-8 jaar, 9-12 jaar, \textgreater12 jaar). Als algemeen
onderzoeksaspect werd het publicatiejaar gecodeerd (2009-2024). Voor het
verband tussen ingrijpende jeugdervaringen en sensitiviteit werd de
effectgrootte, de richting van het verband en de mate van significantie
gecodeerd (zie Bijlage 1).

\subsection{Berekenen effectgroottes}\label{berekenen-effectgroottes}

Uit elk artikel werd de effectgrootte van het verband tussen
sensitiviteit en ingrijpend jeugdtrauma gecodeerd. In eerste instantie
werden alle effectgroottes gecodeerd aan de hand van de Pearson
correlatiecoëfficiënt \(r\). In artikelen waarin deze niet rechtstreeks
te verkrijgen was, werd deze berekend op basis van de beschikbare
gegevens middels een online effectgrootte calculator (Wilson, z.d.). De
Pearson correlatiecoëfficiënt is niet normaal verdeeld en heeft minder
gunstige wiskundige eigenschappen, waardoor er gekozen is om ten behoeve
van de analyses de \(r\)-waarden om te zetten naar \(Fisher’s z\)
waarden (Harrer et al., 2021). Voor eenvoudige interpretatie werden de
\(Fisher’s z\) waarden weer omgerekend naar de Pearson
correlatiecoëfficiënt (Lipsey \& Wilson, 2001). Bij de interpretatie van
de effectgroottes is gekozen voor de interpretatie van Funder en Ozer
(2019). Zij vinden dat effectgroottes worden ondergewaardeerd en onjuist
geïnterpreteerd worden. Ze interpreteren \(r ≥ - .05\) als zeer klein
effect, \(r ≥ -.10\) als een klein effect, \(r ≥ -.20\) een middelgroot
effect, \(r ≥ - .30\) als een groot effect en \(r ≥ -.40\) als een zeer
groot effect. Middels deze richtlijn worden effectgroottes niet meer
genegeerd en adequaat geïnterpreteerd, waardoor ze informatiever zijn
(Funder \& Ozer, 2019). Een negatieve score gaf aan dat een hogere score
op ingrijpende jeugdervaringen bij ouders samenhing met een lagere mate
van sensitiviteit naar hun kind toe.

\subsection{Statistische analyse}\label{statistische-analyse}

Negen studies rapporteerden één effectgrootte. Voor de overige drie
studies is de belangrijkste effectgrootte gekozen, waarbij de
belangrijkste gedefinieerd werd als de meest objectieve gemeten
effectgrootte of de effectgrootte van het gemiddelde van de
steekproefpopulatie. Doordat voor iedere studie één effectgrootte
meegenomen is, hoefde er niet gekeken te worden naar de heterogeniteit
binnen studies en was een drie-level meta-analyse niet noodzakelijk. Er
is daarom gekozen om een metacorrelatie-analyse uit te voeren in een
random-effect-model zodat er meer variantie werd gemodelleerd
(Borenstein et al., 2010). IBM SPSS (Versie 28.0.1.0, IBM Corp, 2021)
werd gebruikt om deze statistische analyses uit te voeren en R (Versie
4.3.3; R Core Team, 2024) werd gebruikt om deze analyses te controleren.
Bij een p-waarde van \textless{} .05 werd het resultaat als statistisch
significant gedefinieerd en er werd een 95\%-betrouwbaarheidsinterval
aangehouden. De heterogeniteit tussen studies werd berekend om zicht te
krijgen op hoe betrouwbaar de effectgroottes zijn en op de varianties
tussen de studieresultaten (Harrer et al., 2021). Om de heterogeniteit
tussen de studies te berekenen werd de Q-statistiek gebruikt (Lipsey \&
Wilson, 2001). Om de betrouwbaarheid van uitspraken over de
heterogeniteit te verhogen werd aanvullend de 𝐼2 statistiek gebruikt
(Harrer et al., 2021). Om deze resultaten te visualiseren werd een
forest plot van de uitkomsten gemaakt (zie Bijlage 2). Tot slot werden
moderatoranalyses uitgevoerd om te onderzoeken of het verband tussen
ingrijpende jeugdervaringen en sensitiviteit beïnvloed wordt door het
aantal ingrijpende jeugdervaringen, de leeftijd van ouders of de
leeftijd van het kind.

\subsection{Publicatiebias}\label{publicatiebias}

Voor huidig onderzoek is gebruik gemaakt van elektronische databases om
studies te vinden die meegenomen konden worden in de meta-analyse. Er
bestaat echter een kans dat niet alle relevante studies geselecteerd
zijn, waardoor er sprake zou kunnen zijn van publicatiebias (Rosenthal,
1995). Dit betekent dat de meegenomen studies in dit onderzoek geen
adequate weergave geven van het werkelijke verband tussen ingrijpende
jeugdervaringen bij ouders en sensitiviteit naar hun kinderen toe.
Mogelijke oorzaken zijn dat studies nog niet in een wetenschappelijk
tijdschrift zijn gepubliceerd, het beoordelingsproces van het artikel
nog lopend is, de resultaten niet significant waren of de resultaten
overeenkwamen met resultaten van eerdere onderzoeken (Maruyama \& Ryan,
2014). Er is door middel van een funnel plot (om symmetrie in de
gegevens zichtbaar te maken), de Egger's test (een kwantitatieve methode
om die symmetrie te testen) en een trim-and-fill analyse (methode om
resultaten vast te stellen op basis van grote studies die gecontroleerd
worden voor kleine studies) bekeken of er mogelijk sprake is van
publicatiebias (Egger et al., 1997).\\
\#\# Resultaten \#\#\# Beschrijvende statistiek In de meta-analyse zijn
twaalf studies geïncludeerd, waarbij er één effectgrootte per studie is
meegenomen die het verband tussen ingrijpende jeugdervaringen van ouders
en de sensitiviteit naar hun kinderen toe beschreef. Zeven van de twaalf
studies namen alle vormen van misbruik en verwaarlozing mee als
ingrijpende jeugdervaring, twee studies namen alleen seksueel misbruik
mee, één studie fysiek misbruik, één studie (echt)scheiding en één
studie nam alle vormen van ingrijpende jeugdervaringen mee. Dit betekent
dat er in vier van de twaalf studies gekeken werd naar één soort
ingrijpende jeugdervaring, in zeven studies naar vijf soorten
ingrijpende jeugdervaringen en in één studie naar alle tien de vormen
van ingrijpende jeugdervaringen. Van de twaalf studies werd in tien
studies alleen gekeken naar moeders en in twee studies naar zowel
moeders als vaders. In vier van de twaalf studies waren de ouders tussen
de 20 en 25 jaar, in twee studies tussen de 26 en 30 jaar, in vijf
studies tussen de 31 en 35 jaar en in één studie werd de leeftijd van
ouders niet gegeven. De kinderen waren in drie van de twaalf studies
jonger dan 6 maanden, in twee studies tussen de 7 en 12 maanden, in zes
studies tussen de 1 en 4 jaar en in één studie tussen de 5 en 8 jaar. De
geïncludeerde studies werden gepubliceerd tussen 2009 en 2024
(\(M = 2018,42, SD = 1,20\)) en de steekproefgrootte varieerde tussen de
15 en 681 (\(M = 184,83, SD = 51,37\)).

\subsection{Overall effect en
heterogeniteit}\label{overall-effect-en-heterogeniteit}

Op basis van de richtlijnen van Funder en Ozer (2019) is er een
middelgroot overall effect geconstateerd tussen ingrijpende
jeugdervaringen bij ouders en sensitiviteit naar hun kinderen
(\(r = -.218, 95% BI = [-0.293, -0.140], p = <.001
\)), wat duidt op een enig verklarend effect dat praktisch nut heeft,
zelf op korte termijn. Dit resultaat wil zeggen dat in 95 van de 100
keer het meemaken van ingrijpende jeugdervaringen bij ouders zorgt voor
een lagere sensitiviteit naar hun kinderen toe. Er bleek verder sprake
te zijn van heterogeniteit tussen studies (\(Q (11) = 29.13, p = .002\))
en uit de \(I^2\) test bleek dat 66,4\% van de totale variantie de
werkelijke verschillen in het populatiegemiddelde weergaf. Er is
daarnaast gekeken, door middel van een reeks beschikbare diagnoses, naar
de aanwezigheid van potentiële uitschieters. Hieruit bleek dat er geen
enkele studie voldoet aan de criteria voor een invloedrijke studie
(Venables \& Smith, 2024).

\subsection{Moderatoren effecten}\label{moderatoren-effecten}

De variantie in alle effectgroottes in de dataset bleek significant
(\(Q (11) = 29.13, p = .002\)), wat betekent dat er een heterogene
verdeling van effectgroottes is. Volgens deze resultaten kunnen
variabelen als moderator getoetst worden voor het gemiddelde effect
(Asscher et al., 2011). Middels moderatoranalyses werd onderzocht of het
aantal ingrijpende jeugdervaringen, de leeftijd van ouders of de
leeftijd van het kind de sterkte van het verband tussen ingrijpende
jeugdervaringen bij ouders en de sensitiviteit op hun kinderen
beïnvloeden (zie Tabel 1). Het aantal ingrijpende jeugdervaringen bleek
geen significant effect te hebben op het verband tussen ingrijpende
jeugdervaringen bij ouders en de sensitiviteit naar hun kinderen toe
(\(Q_{between} (2,9) = .017, p = .991\)). Dit betekent dat het niet
uitmaakt of een ouder één, vijf of tien ingrijpende jeugdervaring(en)
heeft meegemaakt voor het verband tussen ingrijpende jeugdervaringen en
de sensitiviteit naar hun kinderen toe, alhoewel dit resultaat niet
significant is. Daarnaast bleek de leeftijd van ouders geen significant
effect te hebben op het verband tussen ingrijpende jeugdervaringen bij
ouders en de sensitiviteit naar hun kinderen toe
(\(Q_{between} (2,8) = .603, p = .740\)). Voor ouders tussen de 20 en 25
en voor ouders tussen de 26 en 30 geldt dat het verband tussen
ingrijpende jeugdervaringen en sensitiviteit lager is dan voor ouders
tussen de 31 en 35, maar hier bleek alleen het resultaat voor
laatstgenoemde groep significant. Tot slot bleek de leeftijd van het
kind ook geen significant effect te hebben op het verband tussen
ingrijpende jeugdervaringen bij ouders en de sensitiviteit naar hun
kinderen toe (\(Q_{between} (3,8) = .002, p = .999\)). Het verband
tussen ingrijpende jeugdtrauma's en sensitiviteit leek ongeveer gelijk
voor een kind tussen de 0 en 8 jaar, maar deze resultaten bleken niet
significant.\\
Table 1

\subsection{Publicatiebias}\label{publicatiebias-1}

Middels een funnel plot en een Egger's regressie toets werd nagegaan of
er sprake is van publicatiebias. Uit de funnel plot (zie Bijlage 3)
bleek dat er geen sprake is van asymmetrie en er geen aanwijzingen zijn
voor publicatiebias. De Egger's regressie toets
(\(\beta= -.117, p = .230\)) en de funnel plot toets
(\(z = -1.233, p = .061\)) bevestigden dit, aangezien de p-waarden niet
significant waren. Een trim-and-fill analyse was derhalve hier niet
noodzakelijk.

\section{Discussie}\label{discussie}

In de huidige meta-analyse werd het verband onderzocht tussen
ingrijpende jeugdervaringen bij ouders en de sensitiviteit naar hun
kinderen toe. Het doel was om inzicht te krijgen in de richting en
sterkte van dit verband en om te kijken of het aantal ingrijpende
jeugdervaringen, de leeftijd van ouders en de leeftijd van het kind
effect hadden op dit verband. Er is, volgens de richtlijnen van Funder
en Ozer (2019), een middelgroot overall effect gevonden tussen
ingrijpende jeugdervaringen bij ouders en de sensitiviteit naar hun
kinderen toe (\(r = -.218, 95% BI = [-0.293, -0.140], p = <.001
\)). Dit betekent dat ouders die ingrijpende jeugdervaringen hebben
meegemaakt minder sensitief zijn naar hun kinderen toe dan ouders zonder
ingrijpende jeugdervaringen. Dit resultaat komt overeen met de eerste
hypothese. Uit de moderatoranalyses bleek vervolgens dat geen van de
genoemde moderatoren een significant effect had op het verband tussen
ouders die ingrijpende jeugdervaringen hebben meegemaakt en de
sensitiviteit naar hun kinderen toe. Uit eerder onderzoek (Felitti et
al., 1998) kwam naar voren dat bij het meemaken van meerdere vormen
ingrijpende jeugdervaringen de nadelige gevolgen voor de toekomst groter
zouden zijn. In huidig onderzoek kwam dit, tegen de verwachtingen in,
niet naar voren. Dit zou kunnen doordat in sommige onderzoeken de
categorie vier of meer ingrijpende jeugdervaringen werd gehanteerd in
plaats van losse categorieën voor het aantal ingrijpende
jeugdervaringen, waardoor er niet duidelijk was hoeveel ingrijpende
jeugdervaringen er daadwerkelijk zijn meegenomen (Ziv et al., 2018). De
derde hypothese was dat jongere ouders die ingrijpende jeugdervaringen
hebben meegemaakt minder sensitief zouden zijn, maar dit kwam niet uit
de resultaten naar voren. Dit kan mogelijk verklaard worden doordat de
leeftijd van ouders geen voorspeller is, maar er kan ook een andere
verklaring voor zijn. Uit onderzoek bleek dat de leeftijd van ouders
geen effect had op het verband tussen het meemaken van emotioneel
misbruik als ingrijpende jeugdervaringen en de opvoedkwaliteit (Bert et
al., 2009). Uit datzelfde onderzoek bleek dat emotioneel misbruik voor
ouders rond de 35 een minder grote impact had op de kwaliteit van
ouderschap dan fysiek misbruik, terwijl dit niet bleek te gelden voor
tienermoeders. Het kan zijn dat de leeftijd van ouders in combinatie met
het soort ingrijpende jeugdervaring dat zij hebben meegemaakt wel effect
zou hebben op het verband. Tot slot werd verwacht dat ouders met
ingrijpende jeugdervaringen van jongere kinderen minder sensitief zouden
zijn, maar dit werd niet bevestigd door de resultaten. Een verklaring
zou kunnen zijn dat ouders meer moeite krijgen met sensitiviteit naar
hun kind toe als het kind meer autonomie en controle krijgt over zijn
leven, waardoor het lastiger is om sensitief te zijn als ze ouder worden
(Fuchs et al., 2015). In de geïncludeerde studies zijn alleen kinderen
tot en met acht jaar meegenomen. Het zou kunnen dat er wel een effect
gevonden wordt als er in de geïncludeerde studies kinderen tot en met
achttien jaar worden meegenomen, aangezien de autonomie steeds meer
toeneemt gedurende een kind ouder wordt.\\
\#\#\# Sterke punten en beperkingen Een sterk punt van de huidige
meta-analyse is dat er een uitgebreide zoekstring is gebruikt voor het
verzamelen van relevante studies. Een bestaande zoekstring is, in
samenwerking met een Specialist Wetenschappelijke Informatie
Psychologie, Pedagogiek en Onderwijskunde, aangepast en aangevuld op de
huidige meta-analyse. Hierdoor is de kans verkleind dat relevante
artikelen niet zijn opgenomen. Een tweede sterk punt is dat er geen
publicatiebias is vastgesteld, waardoor er vanuit mag worden gegaan dat
de resultaten over het werkelijke verband tussen ingrijpende
jeugdervaringen en sensitiviteit een adequate weergave zijn (Rosenthal,
1995). Ten derde is een sterk punt dat de artikelen en de effectgroottes
van de meeste geïncludeerde studies door meerdere onderzoekers zijn
gecontroleerd, waardoor de betrouwbaarheid van dit onderzoek vergroot
werd (Maruyama \& Ryan, 2014). Tot slot is een sterk punt dat de
meta-analyse heeft geleid tot een grotere statistische power vergeleken
met elk afzonderlijke studie (Maruyama \& Ryan, 2014).\\
Er zijn een aantal beperkingen aan het huidig onderzoek te benoemen. Een
eerste beperking is dat er maar twee studies zijn geïncludeerd waarbij
niet alleen de moeder, maar ook de vader werd onderzocht. De huidige
resultaten zijn daarmee voornamelijk gebaseerd op moeders. De resultaten
zullen met enige voorzichtigheid geïnterpreteerd moeten worden,
aangezien er een kans bestaat dat de resultaten niet generaliseerbaar
zijn naar alle ouders. Een tweede beperking is dat er maar bij één
studie gekeken werd naar alle tien de vormen van ingrijpende
jeugdervaringen. In de meeste studies werden alleen misbruik en
verwaarlozing meegenomen en zijn de ingrijpende jeugdervaringen die
vallen in de categorie `disfunctioneren van het huishouden' niet
meegenomen. Ook daarom zullen de resultaten met enige voorzichtigheid
geïnterpreteerd moeten worden, aangezien er een kans bestaat dat de
huidige resultaten niet gelden voor alle vormen van ingrijpende
jeugdervaringen. Tot slot is er bij de geïncludeerde studies gebruik
gemaakt van zelfrapportage vragenlijsten om de ingrijpende
jeugdervaringen van ouders te meten. Bij zelfrapportage vragenlijsten
kan sprake zijn van responsbias doordat er ruimte is voor eigen
interpretatie. Daarnaast wordt er bij deze vragenlijsten gevraagd naar
gebeurtenissen uit de jeugd waardoor er sprake kan zijn van
geheugenfouten (Maruyama \& Ryan, 2014).

\subsection{Aanbevelingen voor wetenschap en
praktijk}\label{aanbevelingen-voor-wetenschap-en-praktijk}

Ten eerste is het van belang dat er meer onderzoeken worden uitgevoerd
waarin zowel vaders als moeders worden meegenomen, zodat er onderzocht
kan worden of er verschil zit tussen vaders en moeders die ingrijpende
jeugdervaringen hebben meegemaakt en de impact daarvan op de
sensitiviteit naar hun kinderen toe. Als moeders meer tijd brengen met
hun kinderen zorgt dat volgens de roltheorie ervoor dat zij een
accurater begrip hebben van hun signalen (Hallers-Haalboom et al.,
2014). De gender-rollen theorie van Bem (1974) stelt dat de
verschillende kenmerken van vrouwen en mannen kunnen leiden tot een
verschil in opvoeding. Vrouwen zijn competenter in het decoderen van
sociale- en emotionele nonverbale communicatie en daardoor sensitiever
naar hun kinderen toe (Hall \& Matsumoto, 2004). Vader zijn steeds meer
betrokken bij de opvoeding, waardoor het van meerwaarde zou zijn om te
kijken of het verschil tussen mannen en vrouwen in de sensitiviteit nog
altijd aanwezig is. Ten tweede is er ook meer onderzoek nodig naar alle
vormen van ingrijpende jeugdervaringen. Uit eerder onderzoek blijkt dat
het soort ingrijpende jeugdervaring uitmaakt voor de impact op de
kwaliteit van het ouderschap (Bert et al., 2009). Als er meer onderzoek
wordt verricht naar de verschillende soorten ingrijpende
jeugdervaringen, kan er meer gezegd worden over de grootte van de impact
van deze verschillende jeugdervaringen. Er kan dan ook onderzocht worden
of het meemaken van meerdere ingrijpende jeugdervaringen of het meemaken
van een bepaalde soort ingrijpende jeugdervaring een (grotere)
voorspeller is voor het verband tussen ingrijpende jeugdervaringen bij
ouders en sensitiviteit naar hun kinderen toe. Ten derde is het van
belang dat er meer onderzoeken gedaan worden naar ouders van kinderen
tussen de acht en achttien jaar zodat er meer gezegd kan worden over het
verschil tussen jonge en oudere kinderen. Ten slot is het van belang dat
er meer onderzoek wordt verricht naar de emotionele beschikbaarheid,
waarvan in de inleiding al was vastgesteld dat sensitiviteit daar
slechts een onderdeel van is. In eerdere studies is doorgaans alleen
sensitiviteit gemeten, waardoor in huidig onderzoek te weinig informatie
was om emotionele beschikbaarheid als variabele te onderzoeken. Dat
vraagt in de toekomst meer aandacht.

\subsection{Conclusie}\label{conclusie}

Samenvattend, het meemaken van ingrijpende jeugdervaringen heeft een
middelgrote negatieve bijdrage op de sensitiviteit van ouders naar hun
kinderen toe. Het aantal ingrijpende jeugdervaringen, de leeftijd van
ouders en de leeftijd van het kind blijken het verband tussen
ingrijpende jeugdervaringen van ouders en de sensitiviteit naar hun
kinderen toe niet significant te beïnvloeden. Vervolgonderzoek is nodig
om te onderzoeken of deze uitkomsten voor zowel vaders als moeders
gelden en om te bekijken of het soort ingrijpende jeugdervaring effect
heeft op het verband.

\{\{\textless{} pagebreak \textgreater\}\}

\section{Literatuur}\label{literatuur}

Ainsworth, M. D. S., Bell, S. M., \& Stayton, D. J. (1974).
Infant-mother attachment and social development: Socialisation as a
product of reciprocal responsiveness to signals. In M. P. M. Richards
(Ed.), \emph{The introduction of the child into a social world} (pp.~9--
135). London: Cambridge University Press.

Algemeen Nederlands Woordenboek (z.d.). De appel valt niet ver van de
boom. In ANW. Geraadpleegd op 8 maart 2024 van
\url{https://anw.ivdnt.org/}

Alink, L., Prevoo, M., van Berkel, S., Linting, M., Klein Velderman, M.
\& Pannebakker, F. (2018). \emph{NPM 2017: Nationale Prevalentiestudie
Mishandeling van Kinderen en Jeugdigen}. Leiden: Universiteit Leiden/TNO

\textbf{X} \footnote{\textbf{X gebruikt in meta-analyse}} Alto, M. E.,
Warmingham, J. M., Handley, E. D., Rogosch, F., Cicchetti, D., \& Toth,
S. L. (2020). Developmental pathways from maternal history of childhood
maltreatment and maternal depression to toddler attachment and early
childhood behavioral outcomes. \emph{Attachment \& Human Development},
23(3), 328--349. \url{https://doi.org/10.1080/14616734.2020.1734642}

ASReview LAB developers. (2023). \emph{ASReview LAB Software
Documentation (version 1.5).}
Zenodo.\url{https://doi.org/10.5281/zenodo.10066693}

Asscher, J. J., Van Vugt, E. S., Stams, G. J. J., Deković, M.,
Eichelsheim, V. I., \& Yousfi, S. (2011). The relationship between
juvenile psychopathic traits, delinquency and (violent) recidivism: A
meta-analysis. \emph{Journal Of Child Psychology And Psychiatry And
Allied Disciplines}, 52(11), 1134--1143.
\textless{}\url{https://doi.org/10.1111/j.1469-}
7610.2011.02412.x\textgreater{}

Bem, S. L. (1974). The measurement of psychological androgyny.
\emph{Journal Of Consulting And Clinical Psychology}, 42(2), 155--162.
\url{https://doi.org/10.1037/h0036215}

\textbf{X} Bert, S. C., Guner, B. M., \& Lanzi, R. G. (2009). The
Influence of Maternal History of Abuse on Parenting Knowledge and
Behavior. \emph{Family Relations}, 58(2), 176--187.
\url{https://doi.org/10.1111/j.1741-3729.2008.00545.x}

\textbf{X} Bérubé, A., Blais, C., Fournier, A., Turgeon, J., Forget, H.,
Coutu, S., \& Dubeau, D. (2020). Childhood maltreatment moderates the
relationship between emotion recognition and maternal sensitive
behaviors. \emph{Child Abuse \& Neglect}, 102, 104432.
\url{https://doi.org/10.1016/j.chiabu.2020.104432}

\textbf{X} Bérubé, A., Pearson, J., Blais, C., \& Forget, H. (2024).
Stress and emotion recognition predict the relationship between a
history of maltreatment and sensitive parenting behaviors: A
moderated-moderation. \emph{Development And Psychopathology}, 1--11.
\url{https://doi.org/10.1017/s095457942300158x}

Borenstein, M., Hedges, L. V., Higgins, J. P., \& Rothstein, H. R.
(2010). A basic introduction to fixed-effect and random-effects models
for meta-analysis. \emph{Research Synthesis Methods}, 1(2), 97--111.
\url{https://doi.org/10.1002/jrsm.12}

Boullier, M., \& Blair, M. (2018). Adverse childhood experiences.
\emph{Paediatrics And Child Health}, 28(3), 132--137.
\url{https://doi.org/10.1016/j.paed.2017.12.008}

Bowlby, J. (1969). \emph{Attachment and loss. Attachment (Vol. 1)}. New
York: Basic Books.

Brodsky, B. S. (2016). Early childhood environment and genetic
interactions: The diathesis for suicidal behavior. Current
\emph{Psychiatry Reports}, 18(9), 86.

Clark, E., Jiao, Y., Sandoval, K., \& Biringen, Z. (2021).
Neurobiological Implications of Parent--Child Emotional Availability: A
Review. \emph{Brain Sciences}, 11(8), 1016.
\url{https://doi.org/10.3390/brainsci11081016}

Egger, M., Smith, G. D., Schneider, M., \& Minder, C. E. (1997). Bias in
meta-analysis detected by a simple, graphical test. \emph{British
Medical Journal}, 315(7109), 629--634.
\url{https://doi.org/10.1136/bmj.315.7109.629}

Eimers, D. (2021). Intergenerationele overdracht doorbreken. \emph{Augeo
Magazine}, 25.
\url{https://www.augeomagazine.nl/aandacht-voor-ouderschap-augeo-magazine-25/achtergrond-intergenerationele-overdracht-doorbreken}

Elder, G. (1981). History and the life course. In D. Bertaux (Ed.),
\emph{Biography and society: The life history approach in the social
sciences} (pp.77--115). Beverly Hills, CA: Sage.

Felitti, V. J., Anda, R. F., Nordenberg, D., Williamson, D. F., Spitz,
A. M., Edwards, V. J., Koss, M. P., \& Marks, J. S. (1998). Relationship
of Childhood Abuse and Household Dysfunction to Many of the Leading
Causes of Death in Adults. \emph{American Journal Of Preventive
Medicine}, 14(4), 245--258.
\url{https://doi.org/10.1016/s0749-3797(98)00017-8}

\textbf{X} Flagg, A. M., Lin, B., Crnic, K. A., Gonzales, N. A., \&
Luecken, L. J. (2023). Intergenerational Consequences of Maternal
Childhood Maltreatment on Infant Health Concerns. \emph{Maternal And
Child Health Journal}, 27(11), 1981--1989.
\url{https://doi.org/10.1007/s10995-023-03717-1}

\textbf{X} Friesen, M. D., Horwood, L. J., Fergusson, D. M., \&
Woodward, L. J. (2017). Exposure to parental separation in childhood and
later parenting quality as an adult: evidence from a 30‐year
longitudinal study. \emph{Journal Of Child Psychology And Psychiatry And
Allied Disciplines}, 58(1), 30--37.
\url{https://doi.org/10.1111/jcpp.12610}

\textbf{X} Fuchs, A., Möhler, E., Resch, F., \& Kaess, M. (2015). Impact
of a maternal history of childhood abuse on the development of
mother--infant interaction during the first year of life. \emph{Child
Abuse \& Neglect}, 48, 179--189.
\url{https://doi.org/10.1016/j.chiabu.2015.05.023}

Funder, D. C., \& Ozer, D. J. (2019). Evaluating Effect Size in
Psychological Research: Sense and Nonsense. \emph{Advances in Methods
And Practices in Psychological Science}, 2(2), 156--168.
\url{https://doi.org/10.1177/2515245919847202}

Hall, J. A., \& Matsumoto, D. (2004). Gender Differences in Judgments of
Multiple Emotions From Facial Expressions. \emph{Emotion}, 4(2),
201--206. \url{https://doi.org/10.1037/1528-3542.4.2.201}

Hallers-Haalboom, E. T., Mesman, J., Groeneveld, M. G., Endendijk, J.
J., Van Berkel, S. R., Van Der Pol, L. D., \& Bakermans-Kranenburg, M.
J. (2014). Mothers, fathers, sons and daughters: Parental sensitivity in
families with two children. \emph{Journal Of Family Psychology}, 28(2),
138--147. \url{https://doi.org/10.1037/a0036004}

Harrer, M., Cuijpers, P., Furukawa, T. A., \& Ebert, D. D. (2021).
\emph{Doing Meta-Analysis with R}.
\url{https://doi.org/10.1201/9781003107347}

IBM Corp.~Released 2021. \emph{IBM SPSS Statistics for Windows, Version
28.0}. Armonk, NY: IBM Corp

Langevin, R., Marshall, C., \& Kingsland, E. (2019). Intergenerational
Cycles of Maltreatment: A Scoping Review of Psychosocial Risk and
Protective Factors. Trauma, \emph{Violence \& Abuse}, 22(4), 672--688.
\url{https://doi.org/10.1177/1524838019870917}

Lipsey, M.W., \& Wilson, D.B. (2001). \emph{Practical meta-analysis}
(vol.~49). Thousand Oaks, CA: Sage

Lobbestael, G. (2023). \emph{DedupEndNote (Version 1.0.0)}
\[Computer software\]. Geraadpleegd van
\url{https://github.com/globbestael/DedupEndNote}

\textbf{X} MacMillan, K. K., Lewis, A. J., Watson, S. J., Jansen, B., \&
Galbally, M. (2021). Maternal trauma and emotional availability in early
mother-infant interaction: findings from the Mercy Pregnancy and
Emotional Well-being Study (MPEWS) cohort. \emph{Attachment \& Human
Development}, 23(6), 853--875.
\url{https://doi.org/10.1080/14616734.2020.1790116}

Maruyama, G., \& Ryan, C. S. (2014). \emph{Research methods in social
relations}. John Wiley \& Sons.

Narayan, A. J., Lieberman, A. F., \& Masten, A. S. (2021).
Intergenerational transmission and prevention of adverse childhood
experiences (ACEs). \emph{Clinical Psychology Review}, 85, 101997.
\url{https://doi.org/10.1016/j.cpr.2021.101997}

\textbf{X} Olhaberry, M. P., León, M. J., Coo, S., Barrientos, M., \&
Pérez, J. C. (2021). An explanatory model of parental sensitivity in the
mother--father--infant triad. \emph{Infant Mental Health Journal},
43(5), 714--729. \url{https://doi.org/10.1002/imhj.22007}

Page MJ, McKenzie JE, Bossuyt PM, Boutron I, Hoffmann TC, Mulrow CD, et
al.~The PRISMA 2020 statement: an updated guideline for reporting
systematic reviews. \emph{British Medical Journal} 2021;372:n71. doi:
10.1136/bmj.n71

Patterson, G. (1998). Continuities---A search for causal mechanisms:
Comment on the special section. \emph{Developmental Psychology},
34,1263--1268.

\emph{Publication manual of the American Psychological Association (7th
ed.)}. (2020). In American Psychological Association eBooks.
\url{https://doi.org/10.1037/0000165-000}

R Core Team (2024). \emph{R: A language and environment for statistical
computing. R Foundation for Statistical Computing}, Vienna, Austria.
\url{https://www.R-project.org/}

Rosenthal, R. Writing meta-analytic reviews. \emph{Psychological
Bulletin}. 1995, 118, 183--192.

Rowell, T., \& Neal‐Barnett, A. (2021). A Systematic Review of the
Effect of Parental Adverse Childhood Experiences on Parenting and Child
Psychopathology. \emph{Journal Of Child \& Adolescent Trauma}, 15(1),
167--180. \url{https://doi.org/10.1007/s40653-021-00400-x}

Rutter, M. (1998). Some research considerations on intergenerational
continuities and discontinuities: Comment on the special section.
\emph{Developmental Psychology}, 34,1269-- 1273

Savage, L., Tarabulsy, G. M., Pearson, J., Collin‐Vézina, D., \& Gagné,
L. (2019). Maternal history of childhood maltreatment and later
parenting behavior: A meta-analysis. \emph{Development And
Psychopathology}, 31(1), 9--21.
\url{https://doi.org/10.1017/s0954579418001542}

Venables, W. N., \& Smith, D. M. (2024). \emph{An introduction to R:
Notes on R: A Programming environment for data analysis and graphics
Version 4.4.0 (2024-04-24)}.
\url{https://cran.rproject.org/doc/manuals/R-intro.pdf}

Wattanatchariya, K., Narkpongphun, A., \& Kawilapat, S. (2024). The
relationship between parental adverse childhood experiences and
parenting behaviors. \emph{Acta Psychologica}, 243, 104166.
\url{https://doi.org/10.1016/j.actpsy.2024.104166}

Wilson, D. B., Ph.D.~(z.d.). \emph{Practical Meta-Analysis Effect Size
Calculator \[Online calculator\]}. Retrieved 17 april, 2024 from
\url{https://www.campbellcollaboration.org/research-resources/effect-size-calculator.html}

\textbf{X} Ziv, Y., Umphlet, K. L. C., Olarte, S., \& Venza, J. (2018).
Early childhood trauma in highrisk families: associations with caregiver
emotional availability and insightfulness, and children's social
information processing and social behavior. \emph{Attachment \& Human
Development}, 20(3), 309--332.
\url{https://doi.org/10.1080/14616734.2018.1446738}

\emph{Zotero (Version 6.0.36)} \[Computer software\]. (2024)
Geraadpleegd van \url{https://www.zotero.org/}

\textbf{X} Zvara, B. J., Meltzer‐Brody, S., Mills‐Koonce, W. R., \& Cox,
M. (2017). Maternal Childhood Sexual Trauma and Early Parenting:
Prenatal and Postnatal Associations. \emph{Infant And Child
Development}, 26(3). \url{https://doi.org/10.1002/icd.1991}

\textbf{X} Zvara, B. J., Mills-Koonce, W. R., Carmody, K. A., \& Cox, M.
(2015). Childhood sexual trauma and subsequent parenting beliefs and
behaviors. \emph{Child Abuse \& Neglect}, 44, 87--97.
\url{https://doi.org/10.1016/j.chiabu.2015.01.012}

\bibliographystyle{plain}
\bibliography{}



\end{document}
